\setbeamercovered{transparent}
\definecolor{OliveGreen}{rgb}{0,0.6,0}
\definecolor{RoyalBlue}{rgb}{0.254902,0.411765,0.882353}
\definecolor{Maroon}{rgb}{0.698039,0.133333,0.133333}

\subsection{Munkabeosztás és ütemterv}

% % % % % % % % % % % % % % % % % % % % % % % % %
% MUNKAEGYSÉGEK
% % % % % % % % % % % % % % % % % % % % % % % % %
\begin{frame}{Munkabeosztás és ütemterv}
Munkaegységek:
\begin{itemize}
  \item {A kutatás beindítása, adatgyűjtés}
  \item {Az adathalmaz összeállítása}
  \item {Algoritmus kidolgozása}
  \item {Az új és régi módszerek összevetése}
  \item {Az algoritmus optimalizálása, prototípus-fejlesztés}
  \end{itemize}
\end{frame}


% % % % % % % % % % % % % % % % % % % % % % % % %
%RÉSZFELADATOK 1
% % % % % % % % % % % % % % % % % % % % % % % % %
\begin{frame}{Munkabeosztás és ütemterv}
Munkaegységek és részfeladatai:
\begin{itemize}
  \item<1-> {
    A kutatás beindítása, adatgyűjtés
    \begin{enumerate}
	    \item Kutatási módszertani lehetőségek felvázolása
	    \item Kutatási terv kidolgozása
	    \item Meglévő módszerek feltérképezése
	    \item Prototípus első koncepciójának kidolgozása
    \end{enumerate}
  }
  \item<2-> {   
    Az adathalmaz összeállítása
    \begin{enumerate}
        \item Adatok gyűjtése
        \item Standard formátumra való alakítás
        \item Adathalmaz validálása
    \end{enumerate}
  }
  \item<3-> {
    Algoritmus kidolgozása
    \begin{enumerate}
        \item Algoritmus leírása
        \item Teszt jegyzőkönyvek, a statisztikus gépi fordítás hatékonyságával kapcsolatos hatások vizsgálata
    \end{enumerate}
  }
  \end{itemize}
\end{frame}

% % % % % % % % % % % % % % % % % % % % % % % % %
%RÉSZFELADATOK 2
% % % % % % % % % % % % % % % % % % % % % % % % %
\begin{frame}{Munkabeosztás és ütemterv}
Munkaegységek és részfeladatai:
\begin{itemize}
  \item<1-> {
    Az új és régi módszerek összevetése
    \begin{enumerate}
        \item Tesztfuttatások különböző algoritmusokkal és adatokkal, a statisztikus gépi fordítás hatékonyságával kapcsolatos hatások vizsgálata
        \item Az eltérés statisztikus validációja
        \item Szintézis
    \end{enumerate}
  }
  \item<2-> {
    Az algoritmus optimalizálása, prototípus-fejlesztés
    \begin{enumerate}
        \item Performanciaoptimalizált algoritmus
        \item A gépi fordító prototípusának tesztelése és tesztjegyzőkönyvek
        \item Nyilvánosságra hozandó eredmények dokumentációja
        \item A projekt lezárása, dokumentáció
    \end{enumerate}
  }
  \end{itemize}
\end{frame}

% % % % % % % % % % % % % % % % % % % % % % % % %
%GANTT
% % % % % % % % % % % % % % % % % % % % % % % % %
    
\begin{frame}{Munkabeosztás és ütemterv}
	\begin{ganttchart}[
	y unit title=0.4cm,
	y unit chart=0.5cm,
	vgrid,
	title/.append style={draw=none, fill=RoyalBlue!50!black},
	title label font=\tiny\sffamily\bfseries\color{white},
	title label node/.append style={below=-1.6ex},
	title left shift=.05,
	title right shift=-.05,
	title height=1,
	bar/.append style={draw=none, fill=OliveGreen!75},
	bar height=.4,
	bar label font=\tiny\color{black!50},
	group right shift=0,
	group top shift=.4,
	group height=.3,
	group peaks height=.2,
	group label font=\tiny,
	bar incomplete/.append style={fill=Maroon}
	]{1}{14}
	\gantttitle{2nd Semester}{14} \\
	\gantttitlelist{1,...,14}{1} \\
	\ganttset{progress label text={}, link/.style={black, -to}} \\
	\ganttgroup[group label node/.append style={align=right}]{A kutatás beindítása, \ganttalignnewline adatgyűjtés}{1}{7} \\
	\ganttbar[progress=70,bar progress label font=\tiny\color{OliveGreen!75},
		bar progress label node/.append style={right=4pt},
		bar label node/.append style={align=right}]
	{Kutatási módszertani \ganttalignnewline lehetőségek felvázolása}{1}{7} \\
	
	\ganttbar[progress=100,bar progress label font=\tiny\color{OliveGreen!75},
		bar progress label node/.append style={right=4pt},
		bar label node/.append style={align=right}]
	{Kutatási terv kidolgozása}{1}{7} \\
	
	\ganttbar[progress=100,bar progress label font=\tiny\color{OliveGreen!75},
		bar progress label node/.append style={right=4pt},
		bar label node/.append style={align=right}]
	{Meglévő módszerek \ganttalignnewline feltérképezése}{1}{7} \\
	
	\ganttbar[progress=100,bar progress label font=\tiny\color{OliveGreen!75},
		bar progress label node/.append style={right=4pt},
		bar label node/.append style={align=right}]
	{Prototípus első \ganttalignnewline koncepciójának kidolgozása}{1}{7} \\
	
	\ganttgroup[group label node/.append style={align=right}]{Az adathalmaz összeállítása}{1}{7} \\
	\ganttbar[progress=70,bar progress label font=\tiny\color{OliveGreen!75},
		bar progress label node/.append style={right=4pt},
		bar label node/.append style={align=right}]
	{Adatok gyűjtése}{1}{7} \\
		
	\ganttbar[progress=100,bar progress label font=\tiny\color{OliveGreen!75},
		bar progress label node/.append style={right=4pt},
		bar label node/.append style={align=right}]
	{Standard formátumra való \ganttalignnewline alakítás}{1}{7} \\
		
	\ganttbar[progress=100,bar progress label font=\tiny\color{OliveGreen!75},
		bar progress label node/.append style={right=4pt},
		bar label node/.append style={align=right}]
	{Adathalmaz validálása}{1}{7} \\
	
%	\ganttbar[progress=4, name=T1A]{Task A}{2011-01}{2011-06} \\
%	\ganttlinkedbar[progress=0]{Task B}{2011-07}{2011-12} \\
%	\ganttgroup{Objective 2}{2011-01}{2011-12} \\
%	\ganttbar[progress=15, name=T2A]{Task A}{2011-01}{2011-09} \\
%	\ganttlinkedbar[progress=0]{Task B}{2011-10}{2011-12} \\
%	\ganttgroup{Objective 3}{2011-05}{2011-08} \\
%	\ganttbar[progress=0]{Task A}{2011-05}{2011-08}
%	\ganttset{link/.style={OliveGreen}}
%	\ganttlink[link mid=.4]{pp}{T1A}
%	\ganttlink[link mid=.159]{pp}{T2A}
	\end{ganttchart}
\end{frame}

\begin{frame}{Munkabeosztás és ütemterv}
	\begin{ganttchart}[
	y unit title=0.4cm,
	y unit chart=0.6cm,
	vgrid,
	title/.append style={draw=none, fill=RoyalBlue!50!black},
	title label font=\tiny\sffamily\bfseries\color{white},
	title label node/.append style={below=-1.6ex},
	title left shift=.05,
	title right shift=-.05,
	title height=1,
	bar/.append style={draw=none, fill=OliveGreen!75},
	bar height=.4,
	bar label font=\tiny\color{black!50},
	group right shift=0,
	group top shift=.4,
	group height=.3,
	group peaks height=.2,
	group label font=\tiny,
	bar incomplete/.append style={fill=Maroon}
	]{1}{14}
	\gantttitle{2nd Semester}{14} \\
	\gantttitlelist{1,...,14}{1} \\
	\ganttset{progress label text={}, link/.style={black, -to}} \\
	\ganttgroup[group label node/.append style={align=right}]{Algoritmus kidolgozása}{1}{7} \\
	\ganttbar[progress=70,bar progress label font=\tiny\color{OliveGreen!75},
		bar progress label node/.append style={right=4pt},
		bar label node/.append style={align=right}]
	{Algoritmus leírása}{1}{7} \\
	
	\ganttbar[progress=100,bar progress label font=\tiny\color{OliveGreen!75},
		bar progress label node/.append style={right=4pt},
		bar label node/.append style={align=right}]
	{Teszt jegyzőkönyvek,\ganttalignnewline az SMT hatékonyságával\ganttalignnewline kapcsolatos hatások vizsgálata}{1}{7} \\
	
	\ganttgroup[group label node/.append style={align=right}]{Az új és régi módszerek\ganttalignnewline összevetése}{1}{7} \\
	
	\ganttbar[progress=100,bar progress label font=\tiny\color{OliveGreen!75},
		bar progress label node/.append style={right=4pt},
		bar label node/.append style={align=right}]
	{Tesztfuttatások különböző \ganttalignnewline algoritmusokkal}{1}{7} \\
	
	\ganttbar[progress=100,bar progress label font=\tiny\color{OliveGreen!75},
		bar progress label node/.append style={right=4pt},
		bar label node/.append style={align=right}]
	{Az eltérés statisztikus\ganttalignnewline validációja}{1}{7} \\
	
	\ganttbar[progress=70,bar progress label font=\tiny\color{OliveGreen!75},
		bar progress label node/.append style={right=4pt},
		bar label node/.append style={align=right}]
	{Szintézis}{1}{7} \\
	
%	\ganttbar[progress=4, name=T1A]{Task A}{2011-01}{2011-06} \\
%	\ganttlinkedbar[progress=0]{Task B}{2011-07}{2011-12} \\
%	\ganttgroup{Objective 2}{2011-01}{2011-12} \\
%	\ganttbar[progress=15, name=T2A]{Task A}{2011-01}{2011-09} \\
%	\ganttlinkedbar[progress=0]{Task B}{2011-10}{2011-12} \\
%	\ganttgroup{Objective 3}{2011-05}{2011-08} \\
%	\ganttbar[progress=0]{Task A}{2011-05}{2011-08}
%	\ganttset{link/.style={OliveGreen}}
%	\ganttlink[link mid=.4]{pp}{T1A}
%	\ganttlink[link mid=.159]{pp}{T2A}
	\end{ganttchart}
\end{frame}

\begin{frame}{Munkabeosztás és ütemterv}
	\begin{ganttchart}[
	y unit title=0.4cm,
	y unit chart=0.6cm,
	vgrid,
	title/.append style={draw=none, fill=RoyalBlue!50!black},
	title label font=\tiny\sffamily\bfseries\color{white},
	title label node/.append style={below=-1.6ex},
	title left shift=.05,
	title right shift=-.05,
	title height=1,
	bar/.append style={draw=none, fill=OliveGreen!75},
	bar height=.4,
	bar label font=\tiny\color{black!50},
	group right shift=0,
	group top shift=.4,
	group height=.3,
	group peaks height=.2,
	group label font=\tiny,
	bar incomplete/.append style={fill=Maroon}
	]{1}{14}
	\gantttitle{2nd Semester}{14} \\
	\gantttitlelist{1,...,14}{1} \\
	\ganttset{progress label text={}, link/.style={black, -to}} \\
	\ganttgroup[group label node/.append style={align=right}]{Az algoritmus optimalizálása,\ganttalignnewline prototípus-fejlesztés}{1}{7} \\
			
		\ganttbar[progress=100,bar progress label font=\tiny\color{OliveGreen!75},
			bar progress label node/.append style={right=4pt},
			bar label node/.append style={align=right}]
		{Performanciaoptimalizált \ganttalignnewline algoritmus}{1}{7} \\
			
		\ganttbar[progress=100,bar progress label font=\tiny\color{OliveGreen!75},
			bar progress label node/.append style={right=4pt},
			bar label node/.append style={align=right}]
		{A gépi fordító prototípusának \ganttalignnewline tesztelése és tesztjegyzőkönyvek}{1}{7} \\
		
		\ganttbar[progress=100,bar progress label font=\tiny\color{OliveGreen!75},
				bar progress label node/.append style={right=4pt},
				bar label node/.append style={align=right}]
		{Nyilvánosságra hozandó \ganttalignnewline eredmények dokumentációja}{1}{7} \\
		
		\ganttbar[progress=100,bar progress label font=\tiny\color{OliveGreen!75},
				bar progress label node/.append style={right=4pt},
				bar label node/.append style={align=right}]
		{A projekt lezárása,\ganttalignnewline dokumentáció}{1}{7} \\
\end{ganttchart}
\end{frame}