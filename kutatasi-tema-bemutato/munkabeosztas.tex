\setbeamercovered{transparent}

\definecolor{barblue}{RGB}{153,204,254}
\definecolor{groupblue}{RGB}{51,102,254}
\definecolor{linkred}{RGB}{165,0,33}
\renewcommand\sfdefault{phv}
\renewcommand\mddefault{mc}
\renewcommand\bfdefault{bc}
\setganttlinklabel{s-s}{START-TO-START}
\setganttlinklabel{f-s}{FINISH-TO-START}
\setganttlinklabel{f-f}{FINISH-TO-FINISH}
\sffamily

\subsection{Munkabeosztás és ütemterv}

% % % % % % % % % % % % % % % % % % % % % % % % %
% MUNKAEGYSÉGEK
% % % % % % % % % % % % % % % % % % % % % % % % %
\begin{frame}{Munkabeosztás és ütemterv}
Munkaegységek:
\begin{itemize}
  \item {A kutatás beindítása, adatgyűjtés}
  \item {Az adathalmaz összeállítása}
  \item {Algoritmus kidolgozása}
  \item {Az új és régi módszerek összevetése}
  \item {Az algoritmus optimalizálása, prototípus-fejlesztés}
  \end{itemize}
\end{frame}


% % % % % % % % % % % % % % % % % % % % % % % % %
%RÉSZFELADATOK 1
% % % % % % % % % % % % % % % % % % % % % % % % %
\begin{frame}{Munkabeosztás és ütemterv}
Munkaegységek és részfeladatai:
\begin{itemize}
  \item<1-> {
    A kutatás beindítása, adatgyűjtés
    \begin{enumerate}
	    \item Kutatási módszertani lehetőségek felvázolása
	    \item Kutatási terv kidolgozása
	    \item Meglévő módszerek feltérképezése
	    \item Prototípus első koncepciójának kidolgozása
    \end{enumerate}
  }
  \item<2-> {   
    Az adathalmaz összeállítása
    \begin{enumerate}
        \item Adatok gyűjtése
        \item Standard formátumra való alakítás
        \item Adathalmaz validálása
    \end{enumerate}
  }
  \item<3-> {
    Algoritmus kidolgozása
    \begin{enumerate}
        \item Algoritmus leírása
        \item Teszt jegyzőkönyvek, a statisztikus gépi fordítás hatékonyságával kapcsolatos hatások vizsgálata
    \end{enumerate}
  }
  \end{itemize}
\end{frame}

% % % % % % % % % % % % % % % % % % % % % % % % %
%RÉSZFELADATOK 2
% % % % % % % % % % % % % % % % % % % % % % % % %
\begin{frame}{Munkabeosztás és ütemterv}
Munkaegységek és részfeladatai:
\begin{itemize}
  \item<1-> {
    Az új és régi módszerek összevetése
    \begin{enumerate}
        \item Tesztfuttatások különböző algoritmusokkal és adatokkal, a statisztikus gépi fordítás hatékonyságával kapcsolatos hatások vizsgálata
        \item Az eltérés statisztikus validációja
        \item Szintézis
    \end{enumerate}
  }
  \item<2-> {
    Az algoritmus optimalizálása, prototípus-fejlesztés
    \begin{enumerate}
        \item Performanciaoptimalizált algoritmus
        \item A gépi fordító prototípusának tesztelése és tesztjegyzőkönyvek
        \item Nyilvánosságra hozandó eredmények dokumentációja
        \item A projekt lezárása, dokumentáció
    \end{enumerate}
  }
  \end{itemize}
\end{frame}

% % % % % % % % % % % % % % % % % % % % % % % % %
%GANTT
% % % % % % % % % % % % % % % % % % % % % % % % %
\begin{frame}{Munkabeosztás és ütemterv}
	\begin{ganttchart}[
	x unit=0.4cm,
	y unit title=1cm,
	y unit chart=0.5cm,
	    canvas/.append style={fill=none, draw=black!5, line width=.75pt},
	    hgrid style/.style={draw=black!5, line width=.75pt},
	    vgrid={*1{draw=black!5, line width=.75pt}},
	    today=7,
	    today rule/.style={
	      draw=black!64,
	      dash pattern=on 3.5pt off 4.5pt,
	      line width=1.5pt
	    },
	    today label font=\small\bfseries,
	    title/.style={draw=none, fill=none},
	    title label font=\bfseries\footnotesize,
	    title label node/.append style={below=7pt},
	    include title in canvas=false,
	    bar label font=\mdseries\small\color{black!70},
	    bar label node/.append style={left=2cm},
	    bar/.append style={draw=none, fill=black!63},
	    bar incomplete/.append style={fill=barblue},
	    bar progress label font=\mdseries\tiny\color{black!70},
	    group incomplete/.append style={fill=groupblue},
	    group left shift=0,
	    group right shift=0,
	    group height=.5,
	    group peaks tip position=0,
	    group label node/.append style={left=.6cm},
	    group progress label font=\bfseries\tiny,
	    link/.style={-latex, line width=1.5pt, linkred},
	    link label font=\scriptsize\bfseries,
	    link label node/.append style={below left=-2pt and 0pt}
	  ]{1}{13}
	  \gantttitle[
	    title label node/.append style={below left=7pt and -3pt}
	  ]{HÉT:\quad1}{1}
	  \gantttitlelist{2,...,14}{1} \\
	  \ganttgroup[progress=57]{WBS 1 Summary Element 1}{1}{10} \\
	  \ganttbar[
	    progress=75,
	    name=WBS1A
	  ]{\textbf{WBS 1.1} Activity A}{1}{8} \\
	  \ganttbar[
	    progress=67,
	    name=WBS1B
	  ]{\textbf{WBS 1.2} Activity B}{1}{3} \\
	  \ganttbar[
	    progress=50,
	    name=WBS1C
	  ]{\textbf{WBS 1.3} Activity C}{4}{10} \\
	  \ganttbar[
	    progress=0,
	    name=WBS1D
	  ]{\textbf{WBS 1.4} Activity D}{4}{10} \\[grid]
	  \ganttgroup[progress=0]{WBS 2 Summary Element 2}{4}{10} \\
	  \ganttbar[progress=0]{\textbf{WBS 2.1} Activity E}{4}{5} \\
	  \ganttbar[progress=0]{\textbf{WBS 2.2} Activity F}{6}{8} \\
	  \ganttbar[progress=0]{\textbf{WBS 2.3} Activity G}{9}{10}
%	  \ganttlink[link type=s-s]{WBS1A}{WBS1B}
%	  \ganttlink[link type=f-s]{WBS1B}{WBS1C}
%	  \ganttlink[
%	    link type=f-f,
%	    link label node/.append style=left
%	  ]{WBS1C}{WBS1D}
	\end{ganttchart}
\end{frame}