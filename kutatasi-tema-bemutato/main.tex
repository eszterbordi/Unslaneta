% Copyright 2004 by Till Tantau <tantau@users.sourceforge.net>.
%
% In principle, this file can be redistributed and/or modified under
% the terms of the GNU Public License, version 2.
%
% However, this file is supposed to be a template to be modified
% for your own needs. For this reason, if you use this file as a
% template and not specifically distribute it as part of a another
% package/program, I grant the extra permission to freely copy and
% modify this file as you see fit and even to delete this copyright
% notice. 

\documentclass{beamer}

\usepackage{docmute}
\usepackage{pgfgantt}

\usepackage[utf8]{inputenc}
\usepackage[hungarian]{babel}

% There are many different themes available for Beamer. A comprehensive
% list with examples is given here:
% http://deic.uab.es/~iblanes/beamer_gallery/index_by_theme.html
% You can uncomment the themes below if you would like to use a different
% one:

\usetheme{Hannover}


\title{Kell ide egy cím}

% A subtitle is optional and this may be deleted
\subtitle{A statisztikus gépi fordítás hatékonyságának jelentés egyértelműsítéssel történő javítása}

\author{E.~Bordi, T.~Both, Sz.~Pável, Cs.~Sándor, A.~Szász}
% - Give the names in the same order as the appear in the paper.
% - Use the \inst{?} command only if the authors have different
%   affiliation.

\institute[Babe\c{s}-Bolyai Tudományegyetem] % (optional, but mostly needed)
{Babe\c{s}-Bolyai Tudományegyetem, Matematika és Informatika Kar, Kolozsvár}
% - Use the \inst command only if there are several affiliations.
% - Keep it simple, no one is interested in your street address.

\date{2016 április 15.}
% - Either use conference name or its abbreviation.
% - Not really informative to the audience, more for people (including
%   yourself) who are reading the slides online 

% If you have a file called "university-logo-filename.xxx", where xxx
% is a graphic format that can be processed by latex or pdflatex,
% resp., then you can add a logo as follows:

% \pgfdeclareimage[height=0.5cm]{university-logo}{university-logo-filename}
% \logo{\pgfuseimage{university-logo}}

% Let's get started
\begin{document}


% Sections are stored in separate .tex files
\input{cimoldal}
\begin{frame}{Tartalom}
  \tableofcontents
  % You might wish to add the option [pausesections]
\end{frame}
\section{Motiváció}
\begin{frame}{Motiváció}
\begin{itemize}
  \item {Szövegek fordítása a számítógép által}
  \item {Statisztikus gépi fordítás (SMT)}
  \item {Filmfeliratok használata tanulási adatként}
  \item {Hatékonyságon javítani}
  \item {Jelentés egyértelműsítése (WSD)}
  
\end{itemize}

\end{frame}

\section{Létező megoldások}
\section{Létező megoldások}
\begin{frame}{Létező megoldások}
\framesubtitle{Statistical machine translation}
	
	\begin{figure}[t]
		\includegraphics[width=1\linewidth]{images/smt_translation}
 	\end{figure}

	\begin{equation}
		P(e|f) = \frac{P(e)P(f|e)}{P(f)}
	\end{equation}
	\begin{equation}
		T(e) = \hat{e} = argmax_e P(e|f) = argmax_e P(e) P(f|e)
	\end{equation}
	
	
\end{frame}

\begin{frame}{Létező megoldások}
\framesubtitle{Statistical machine translation}

	\begin{align*}
		T(e) = \hat{e} = argmax_e P(e) P(f|e)
	\end{align*}
	
	\begin{itemize}

		\item
			nyelvi modell: $P(e)$
			\begin{itemize}
				\item
					folytonosságot biztosít a célnyelvben
			\end{itemize}

		\item
			fordítási modell: $P(f|e)$
				\begin{itemize}
					\item
			 			lexikális megfeleltetés a nyelvek között
			 		\item
			 			szó alapú (word based) modellek \cite{Brown:1990} \cite{Berger:1994}
			 		\item
			 			kifejezés alapú (phrase based) modellek \cite{Och99improvedalignment} \cite{Marcu:2002}
			 	\end{itemize}
		\item
			argmax: $argmax_e$
			\begin{itemize}
				\item
					keresés
			\end{itemize}
	\end{itemize}
	
	
\end{frame}

\begin{frame}{Létező megoldások}
	\framesubtitle{Előnyök, hátrányok}
	
	Szó alapú vs kifejezés alapú modellek: 
	\begin{itemize}
		\item
			szó alapú modell: nehéz a tokeninzálás \cite{Lopez07asurvey}
		\item
			kifejezés alapú modell pontosabb komplex nyelveknél \cite{Lopez07asurvey}
		
	\end{itemize}
	
	Hátrányai ezeknek a rendszereknek:
	\begin{itemize}
		\item
			Még mindig nem elég pontosak az SMT rendszerek a WSD-hez képest \cite{Carpuat_evaluatingthe}
	\end{itemize}
	
\end{frame}


\begin{frame}{Létező megoldások}
	\framesubtitle{Aktuális próbálkozások a javításra}
	
	\begin{itemize}
		\item
			WSD beépítése az SMT-be: \cite{carpuat2005} \cite{Carpuat07improvingstatistical}
	\end{itemize}
\end{frame}




\section{Kutatási terv}
\subsection{Megtenni szándékozott lépések}

\begin{frame}
	\begin{itemize}
		\item Jelenleg legjobban teljesítő algoritmusok megtalálása
		\begin{enumerate}
			\item Statistical Machine Translation módszerek keresése
			\item SMT algoritmusok telesítményének feltérképezése
		\end{enumerate}
		
		\item Jelentés egyérelműsítés alkalmazása
		\begin{enumerate}
			\item Jelentésbeli közelséget vizsgáló módszerek felkutatása
			\item Modellek beépítése SMT módszerekbe
		\end{enumerate}
		
		\item Tanuló és teszt adathalmaz összeállítása
		\begin{enumerate}
			\item Párosított mondatok adatbázisának felépítése
			\item Lehetséges források: filmfeliratok, TED talk feliratok
			\item Szemantikai közelség vizsgálatához Wikipedia cikkek indexelése
		\end{enumerate}
		
	\end{itemize}
\end{frame}

\begin{frame}
	\begin{itemize}
	\item Kiértékelési módszertan meghatározása
		\begin{enumerate}
			\item Használt metrikák meghatározása
			\item Automata kiértékelés szerkesztése
			\item Módszer validálása manuális kiértékelés által
		\end{enumerate}
		
		\item Prototípus elkészítése
		\begin{enumerate}
			\item Algoritmus implementálása
			\item Algoritmus alkalmazása valós környezetben
		\end{enumerate}
	\end{itemize}
\end{frame}
\subsection{Csapat tagjainak hozzájárulása}

\begin{frame}{Csapat tagjainak hozzájárulása}

\begin{table}
\begin{tabular}{l | c }
Feladatkör & Csapattag\\
\hline \hline
Megvalósítandó anyagok & {\color{blue}Bordi Eszter}\\
Munkabeosztás & \\
Gantt diagram & \\
Motiváció & {\color{blue}Both Tibor}\\ 
Megtenni szándékozott lépések & {\color{blue}Pável Szabolcs}\\
Létező megoldások & {\color{blue}Sándor Csanád}\\
Csapat tagjainak hozzájárulása & {\color{blue}Szász Adorján}
\end{tabular}
\end{table}

\end{frame}
\subsection{Megvalósítandó anyagok}
\begin{frame}{Megvalósítandó anyagok}

\end{frame}
\setbeamercovered{transparent}
\definecolor{OliveGreen}{rgb}{0,0.6,0}
\definecolor{RoyalBlue}{rgb}{0.254902,0.411765,0.882353}
\definecolor{Maroon}{rgb}{0.698039,0.133333,0.133333}

\subsection{Munkabeosztás és ütemterv}

% % % % % % % % % % % % % % % % % % % % % % % % %
% MUNKAEGYSÉGEK
% % % % % % % % % % % % % % % % % % % % % % % % %
\begin{frame}{Munkabeosztás és ütemterv}
Munkaegységek:
\begin{itemize}
  \item {A kutatás beindítása, adatgyűjtés}
  \item {Az adathalmaz összeállítása}
  \item {Algoritmus kidolgozása}
  \item {Az új és régi módszerek összevetése}
  \item {Az algoritmus optimalizálása, prototípus-fejlesztés}
  \end{itemize}
\end{frame}


% % % % % % % % % % % % % % % % % % % % % % % % %
%RÉSZFELADATOK 1
% % % % % % % % % % % % % % % % % % % % % % % % %
\begin{frame}{Munkabeosztás és ütemterv}
Munkaegységek és részfeladatai:
\begin{itemize}
  \item<1-> {
    A kutatás beindítása, adatgyűjtés
    \begin{enumerate}
	    \item Kutatási módszertani lehetőségek felvázolása
	    \item Kutatási terv kidolgozása
	    \item Meglévő módszerek feltérképezése
	    \item Prototípus első koncepciójának kidolgozása
    \end{enumerate}
  }
  \item<2-> {   
    Az adathalmaz összeállítása
    \begin{enumerate}
        \item Adatok gyűjtése
        \item Standard formátumra való alakítás
        \item Adathalmaz validálása
    \end{enumerate}
  }
  \item<3-> {
    Algoritmus kidolgozása
    \begin{enumerate}
        \item Algoritmus leírása
        \item Teszt jegyzőkönyvek, a statisztikus gépi fordítás hatékonyságával kapcsolatos hatások vizsgálata
    \end{enumerate}
  }
  \end{itemize}
\end{frame}

% % % % % % % % % % % % % % % % % % % % % % % % %
%RÉSZFELADATOK 2
% % % % % % % % % % % % % % % % % % % % % % % % %
\begin{frame}{Munkabeosztás és ütemterv}
Munkaegységek és részfeladatai:
\begin{itemize}
  \item<1-> {
    Az új és régi módszerek összevetése
    \begin{enumerate}
        \item Tesztfuttatások különböző algoritmusokkal és adatokkal, a statisztikus gépi fordítás hatékonyságával kapcsolatos hatások vizsgálata
        \item Az eltérés statisztikus validációja
        \item Szintézis
    \end{enumerate}
  }
  \item<2-> {
    Az algoritmus optimalizálása, prototípus-fejlesztés
    \begin{enumerate}
        \item Performanciaoptimalizált algoritmus
        \item A gépi fordító prototípusának tesztelése és tesztjegyzőkönyvek
        \item Nyilvánosságra hozandó eredmények dokumentációja
        \item A projekt lezárása, dokumentáció
    \end{enumerate}
  }
  \end{itemize}
\end{frame}

% % % % % % % % % % % % % % % % % % % % % % % % %
%GANTT
% % % % % % % % % % % % % % % % % % % % % % % % %
    
\begin{frame}{Munkabeosztás és ütemterv}
	\begin{ganttchart}[
	y unit title=0.4cm,
	y unit chart=0.5cm,
	vgrid,
	title/.append style={draw=none, fill=RoyalBlue!50!black},
	title label font=\tiny\sffamily\bfseries\color{white},
	title label node/.append style={below=-1.6ex},
	title left shift=.05,
	title right shift=-.05,
	title height=1,
	bar/.append style={draw=none, fill=OliveGreen!75},
	bar height=.4,
	bar label font=\tiny\color{black!50},
	group right shift=0,
	group top shift=.4,
	group height=.3,
	group peaks height=.2,
	group label font=\tiny,
	bar incomplete/.append style={fill=Maroon}
	]{1}{14}
	\gantttitle{2nd Semester}{14} \\
	\gantttitlelist{1,...,14}{1} \\
	\ganttset{progress label text={}, link/.style={black, -to}} \\
	\ganttgroup[group label node/.append style={align=right}]{A kutatás beindítása, \ganttalignnewline adatgyűjtés}{1}{7} \\
	\ganttbar[progress=70,bar progress label font=\tiny\color{OliveGreen!75},
		bar progress label node/.append style={right=4pt},
		bar label node/.append style={align=right}]
	{Kutatási módszertani \ganttalignnewline lehetőségek felvázolása}{1}{7} \\
	
	\ganttbar[progress=100,bar progress label font=\tiny\color{OliveGreen!75},
		bar progress label node/.append style={right=4pt},
		bar label node/.append style={align=right}]
	{Kutatási terv kidolgozása}{1}{7} \\
	
	\ganttbar[progress=100,bar progress label font=\tiny\color{OliveGreen!75},
		bar progress label node/.append style={right=4pt},
		bar label node/.append style={align=right}]
	{Meglévő módszerek \ganttalignnewline feltérképezése}{1}{7} \\
	
	\ganttbar[progress=100,bar progress label font=\tiny\color{OliveGreen!75},
		bar progress label node/.append style={right=4pt},
		bar label node/.append style={align=right}]
	{Prototípus első \ganttalignnewline koncepciójának kidolgozása}{1}{7} \\
	
	\ganttgroup[group label node/.append style={align=right}]{Az adathalmaz összeállítása}{1}{7} \\
	\ganttbar[progress=70,bar progress label font=\tiny\color{OliveGreen!75},
		bar progress label node/.append style={right=4pt},
		bar label node/.append style={align=right}]
	{Adatok gyűjtése}{1}{7} \\
		
	\ganttbar[progress=100,bar progress label font=\tiny\color{OliveGreen!75},
		bar progress label node/.append style={right=4pt},
		bar label node/.append style={align=right}]
	{Standard formátumra való \ganttalignnewline alakítás}{1}{7} \\
		
	\ganttbar[progress=100,bar progress label font=\tiny\color{OliveGreen!75},
		bar progress label node/.append style={right=4pt},
		bar label node/.append style={align=right}]
	{Adathalmaz validálása}{1}{7} \\
	
%	\ganttbar[progress=4, name=T1A]{Task A}{2011-01}{2011-06} \\
%	\ganttlinkedbar[progress=0]{Task B}{2011-07}{2011-12} \\
%	\ganttgroup{Objective 2}{2011-01}{2011-12} \\
%	\ganttbar[progress=15, name=T2A]{Task A}{2011-01}{2011-09} \\
%	\ganttlinkedbar[progress=0]{Task B}{2011-10}{2011-12} \\
%	\ganttgroup{Objective 3}{2011-05}{2011-08} \\
%	\ganttbar[progress=0]{Task A}{2011-05}{2011-08}
%	\ganttset{link/.style={OliveGreen}}
%	\ganttlink[link mid=.4]{pp}{T1A}
%	\ganttlink[link mid=.159]{pp}{T2A}
	\end{ganttchart}
\end{frame}

\begin{frame}{Munkabeosztás és ütemterv}
	\begin{ganttchart}[
	y unit title=0.4cm,
	y unit chart=0.6cm,
	vgrid,
	title/.append style={draw=none, fill=RoyalBlue!50!black},
	title label font=\tiny\sffamily\bfseries\color{white},
	title label node/.append style={below=-1.6ex},
	title left shift=.05,
	title right shift=-.05,
	title height=1,
	bar/.append style={draw=none, fill=OliveGreen!75},
	bar height=.4,
	bar label font=\tiny\color{black!50},
	group right shift=0,
	group top shift=.4,
	group height=.3,
	group peaks height=.2,
	group label font=\tiny,
	bar incomplete/.append style={fill=Maroon}
	]{1}{14}
	\gantttitle{2nd Semester}{14} \\
	\gantttitlelist{1,...,14}{1} \\
	\ganttset{progress label text={}, link/.style={black, -to}} \\
	\ganttgroup[group label node/.append style={align=right}]{Algoritmus kidolgozása}{1}{7} \\
	\ganttbar[progress=70,bar progress label font=\tiny\color{OliveGreen!75},
		bar progress label node/.append style={right=4pt},
		bar label node/.append style={align=right}]
	{Algoritmus leírása}{1}{7} \\
	
	\ganttbar[progress=100,bar progress label font=\tiny\color{OliveGreen!75},
		bar progress label node/.append style={right=4pt},
		bar label node/.append style={align=right}]
	{Teszt jegyzőkönyvek,\ganttalignnewline az SMT hatékonyságával\ganttalignnewline kapcsolatos hatások vizsgálata}{1}{7} \\
	
	\ganttgroup[group label node/.append style={align=right}]{Az új és régi módszerek\ganttalignnewline összevetése}{1}{7} \\
	
	\ganttbar[progress=100,bar progress label font=\tiny\color{OliveGreen!75},
		bar progress label node/.append style={right=4pt},
		bar label node/.append style={align=right}]
	{Tesztfuttatások különböző \ganttalignnewline algoritmusokkal}{1}{7} \\
	
	\ganttbar[progress=100,bar progress label font=\tiny\color{OliveGreen!75},
		bar progress label node/.append style={right=4pt},
		bar label node/.append style={align=right}]
	{Az eltérés statisztikus\ganttalignnewline validációja}{1}{7} \\
	
	\ganttbar[progress=70,bar progress label font=\tiny\color{OliveGreen!75},
		bar progress label node/.append style={right=4pt},
		bar label node/.append style={align=right}]
	{Szintézis}{1}{7} \\
	
%	\ganttbar[progress=4, name=T1A]{Task A}{2011-01}{2011-06} \\
%	\ganttlinkedbar[progress=0]{Task B}{2011-07}{2011-12} \\
%	\ganttgroup{Objective 2}{2011-01}{2011-12} \\
%	\ganttbar[progress=15, name=T2A]{Task A}{2011-01}{2011-09} \\
%	\ganttlinkedbar[progress=0]{Task B}{2011-10}{2011-12} \\
%	\ganttgroup{Objective 3}{2011-05}{2011-08} \\
%	\ganttbar[progress=0]{Task A}{2011-05}{2011-08}
%	\ganttset{link/.style={OliveGreen}}
%	\ganttlink[link mid=.4]{pp}{T1A}
%	\ganttlink[link mid=.159]{pp}{T2A}
	\end{ganttchart}
\end{frame}

\begin{frame}{Munkabeosztás és ütemterv}
	\begin{ganttchart}[
	y unit title=0.4cm,
	y unit chart=0.6cm,
	vgrid,
	title/.append style={draw=none, fill=RoyalBlue!50!black},
	title label font=\tiny\sffamily\bfseries\color{white},
	title label node/.append style={below=-1.6ex},
	title left shift=.05,
	title right shift=-.05,
	title height=1,
	bar/.append style={draw=none, fill=OliveGreen!75},
	bar height=.4,
	bar label font=\tiny\color{black!50},
	group right shift=0,
	group top shift=.4,
	group height=.3,
	group peaks height=.2,
	group label font=\tiny,
	bar incomplete/.append style={fill=Maroon}
	]{1}{14}
	\gantttitle{2nd Semester}{14} \\
	\gantttitlelist{1,...,14}{1} \\
	\ganttset{progress label text={}, link/.style={black, -to}} \\
	\ganttgroup[group label node/.append style={align=right}]{Az algoritmus optimalizálása,\ganttalignnewline prototípus-fejlesztés}{1}{7} \\
			
		\ganttbar[progress=100,bar progress label font=\tiny\color{OliveGreen!75},
			bar progress label node/.append style={right=4pt},
			bar label node/.append style={align=right}]
		{Performanciaoptimalizált \ganttalignnewline algoritmus}{1}{7} \\
			
		\ganttbar[progress=100,bar progress label font=\tiny\color{OliveGreen!75},
			bar progress label node/.append style={right=4pt},
			bar label node/.append style={align=right}]
		{A gépi fordító prototípusának \ganttalignnewline tesztelése és tesztjegyzőkönyvek}{1}{7} \\
		
		\ganttbar[progress=100,bar progress label font=\tiny\color{OliveGreen!75},
				bar progress label node/.append style={right=4pt},
				bar label node/.append style={align=right}]
		{Nyilvánosságra hozandó \ganttalignnewline eredmények dokumentációja}{1}{7} \\
		
		\ganttbar[progress=100,bar progress label font=\tiny\color{OliveGreen!75},
				bar progress label node/.append style={right=4pt},
				bar label node/.append style={align=right}]
		{A projekt lezárása,\ganttalignnewline dokumentáció}{1}{7} \\
\end{ganttchart}
\end{frame}


% Section and subsections will appear in the presentation overview
% and table of contents.
%\section{First Main Section}
%
%\subsection{First Subsection}
%
%\begin{frame}{First Slide Title}{Optional Subtitle}
%  \begin{itemize}
%  \item {
%    My first point.
%  }
%  \item {
%    My second point.
%  }
%  \end{itemize}
%\end{frame}
%
%\subsection{Second Subsection}
%
%% You can reveal the parts of a slide one at a time
%% with the \pause command:
%\begin{frame}{Second Slide Title}
%  \begin{itemize}
%  \item {
%    First item.
%    \pause % The slide will pause after showing the first item
%  }
%  \item {   
%    Second item.
%  }
%  % You can also specify when the content should appear
%  % by using <n->:
%  \item<3-> {
%    Third item.
%  }
%  \item<4-> {
%    Fourth item.
%  }
%  % or you can use the \uncover command to reveal general
%  % content (not just \items):
%  \item<5-> {
%    Fifth item. \uncover<6->{Extra text in the fifth item.}
%  }
%  \end{itemize}
%\end{frame}
%
%\section{Second Main Section}
%
%\subsection{Another Subsection}
%
%\begin{frame}{Blocks}
%\begin{block}{Block Title}
%You can also highlight sections of your presentation in a block, with it's own title
%\end{block}
%\begin{theorem}
%There are separate environments for theorems, examples, definitions and proofs.
%\end{theorem}
%\begin{example}
%Here is an example of an example block.
%\end{example}
%\end{frame}
%
%% Placing a * after \section means it will not show in the
%% outline or table of contents.
%\section*{Summary}
%
%\begin{frame}{Summary}
%  \begin{itemize}
%  \item
%    The \alert{first main message} of your talk in one or two lines.
%  \item
%    The \alert{second main message} of your talk in one or two lines.
%  \item
%    Perhaps a \alert{third message}, but not more than that.
%  \end{itemize}
%  
%  \begin{itemize}
%  \item
%    Outlook
%    \begin{itemize}
%    \item
%      Something you haven't solved.
%    \item
%      Something else you haven't solved.
%    \end{itemize}
%  \end{itemize}
%\end{frame}



% All of the following is optional and typically not needed. 
\appendix
\section<presentation>*{\appendixname}
\subsection<presentation>*{Irodalomjegyzék}

\begin{frame}[allowframebreaks]
  \frametitle<presentation>{Irodalomjegyzék}
    
  \begin{thebibliography}{10}
    
  \beamertemplatebookbibitems
  % Start with overview books.

  \bibitem{Author1990}
    A.~Author.
    \newblock {\em Handbook of Everything}.
    \newblock Some Press, 1990.
 
    
  \beamertemplatearticlebibitems
  % Followed by interesting articles. Keep the list short. 

  \bibitem{Someone2000}
    S.~Someone.
    \newblock On this and that.
    \newblock {\em Journal of This and That}, 2(1):50--100,
    2000.
  \end{thebibliography}
\end{frame}

\end{document}


