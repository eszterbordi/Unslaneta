\section{Bevezetés}
Gépi fordítást elsősorban szövegek megértésének elősegítésére, információszerzésre használják, jelenleg nem alkalmas emberi fordítással azonos minőségű fordítás létrehozására. A statisztikus gepi forditas ennek egyik ága, amely a fordítást statisztikai modellek segítségevel végzi.
A legfőbb nehézséget a többértelmű szavak, valmint a kifejezések fordítása okozza. Mivel a statisztikai modellek nem veszik figyelmebe a szavak többértelműségét, ezért beépítettük a szövegegyértelműsítést rendszerünkbe. A wikipedia adathalmaza lett felhasználva, mivel az egyik legnagyobb létező adathalmaz, amely elérhető több nyelven is. 

Ebben a cikkben elemezzük eredményeinket, melyből látható, hogy a szövegegyértelműsítés használatával növelhető a statisztikus gépi fordítás hatékonysága.