\section{Bevezetés}
Gépi fordítást elsősorban szövegek megértésének elősegítésére, információszerzésre használnak, jelenleg nem alkalmas emberi fordítással azonos minőségű fordítás létrehozására. A statisztikus gépi fordítás ennek az egyik ága, amely a fordítást statisztikai modellek segítségevel végzi.
A legfőbb nehézséget a többértelmű szavak, valamint a kifejezések fordítása okozza.
Számottevő fejlődést a statisztikus gépi fordítás számára az jelentett, hogy a modelleket szavak helyett a szókapcsolatokra építették. Azonban ezek a modellek még mindig nem veszik figyelembe a szavak/kifejezések többértelműségét, ezért kiegészítettük ezt a jelentés-egyértelműsítés módszerével is. A Wikipedia adathalmaza - az egyik legnagyobb létező - lett felhasználva. Ez elérhető több nyelven is. 

Ebben a cikkben elemezzük eredményeinket, melyből látható, hogy a szövegegyértelműsítés használatával növelhető a statisztikus gépi fordítás hatékonysága, pontossága.