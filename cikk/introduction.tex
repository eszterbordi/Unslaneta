\section{Bevezetés}
Gépi fordítást elsősorban szövegek megértésének elősegítésére, információszerzésre használják, jelenleg nem alkalmas emberi fordítással azonos minőségű fordítás létrehozására. A statisztikus gepi forditas ennek egyik ága, amely a fordítást statisztikai modellek segítségevel végzi.
A legfőbb nehézséget a többértelmű szavak, valmint a kifejezések fordítása okozza.
A legnagyobb fejlődést  statisztikus gepi forditas számára az jelentette, amikor a modelleket szavak helyett a szókapcsolatokra kezdték építeni. Azonban ezek a modellek még mindig nem veszik figyelmebe a szavak/kifejezések többértelműségét, ezért hozzáadtuk a szövegegyértelműsítés módszerét is. A wikipedia adathalmaza lett felhasználva, mivel az egyik legnagyobb létező adathalmaz, amely elérhető több nyelven is. 

Ebben a cikkben elemezzük eredményeinket, melyből látható, hogy a szövegegyértelműsítés használatával növelhető a statisztikus gépi fordítás hatékonysága.