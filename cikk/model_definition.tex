\section{A modell definiálása és letező megoldások} \label{sec:model_definition}

\subsection{A statisztikus gépi fordítás}

A statisztikus módszerek használatát a gépi fordításban legelőször Warren Weaver javasolta egy levélben 1949-ben. Azonban az elméleti nehézségeknek és az akkori számítógépek fejlettségének tulajdoníthatóan a módszert csak 1990-ben próbálták ki \cite{Brown:1990:SAM:92858.92860}.

Az általunk felhasznált modell -- a kifejezés alapú fordító modell -- a zajos csatorna modellen alapszik -- \ref{fig:modell:fig1}. Tételezzük fel, hogy egy $F$ nyelv valamilyen $f$ szövegének fordítása egy $E$ célnyelvbe az $e$-vel jelölt szöveg. Annak valószínűsége, hogy $f$-et lefordítva $e$-t kapunk: $P(e|f)$. Felhasználva Bayes képletét a feltételes valószínűség a következő alakba írható:

\begin{equation}
	P(e|f) = \frac{P(e)P(f|e)}{P(f)}
\end{equation}

A nevezőt elhagyva a következő optimalizálási feladathoz jutunk:

\begin{equation}
	T(f) = argmax_e P(e|f) = argmax_e P(e)P(f|e)
\end{equation} 

Ezzel a feladatunk három részre osztható: 
\begin{itemize}
	\item
		A nyelvi modell --  $P(e)$ -- mely a célnyelv folytonosságát biztosítja.
	\item
		A fordítási modell --  $P(f|e)$ -- mellyel növelhető a lexikális megfeleltetések pontossága.
	\item
		A dekóder -- $argmax$ -- kiválasztja a legjobban megfelelő $e$ szöveget a célnyelvből.
\end{itemize}

A fordítási modell első változatai a szó alapú modellek -- IBM modell, melyek nem veszik figyelembe a szövegkontextust. A modellben a forrás $f$ és cél $e$ szövegeknek csak különálló szavak felelnek meg, így „word-to-word” fordítást valósítanak meg. A modell azt vizsgálja, hogy egy adott $f$ szónak milyen valószínűséggel felel meg a célnyelv $e$ szava \cite{Brown:1990:SAM:92858.92860} \cite{Berger:1994}.

A szó alapú modelleknél jobb megoldások a kifejezés alapú modellek \cite{Marcu:2002}, \cite{Och99improvedalignment}. A szó alapú modellek nagy hátránya a bonyolult tokenizálás \cite{Lopez07asurvey}, valamint komplex nyelvek esetén a fordítási pontatlanság \cite{Lopez07asurvey}. A kifejezés alapú modell a szavak helyett $(f_i, e_i)$ kifejezés párokat vizsgál. A $P(f_i|e_i)$ feltételes valószínűséget a korpuszból frekvenciaszámmal becsüljük. A lehetséges kifejezéspárokból és hozzájuk tartozó feltételes valószínűségekből felépítjük a fordítási táblát, melyet felhasználunk a fordítási modellhez.










\begin{figure}[b]
  	\centering
  		\pgfimage[width=0.6\linewidth]{images/smt_translation_model}
  	\caption[smt_translation_model]%
  	{A zajos csatorna modell}
 	\label{fig:modell:fig1}
\end{figure}





\subsection{Jelentés-egyértelműsítés}
bemutatni a wsd-t?

jelenlegi phrase based megoldasok
	miert jok, miert nem
	osszehasonlitasok mas modszerekkel
milyen probalkozasok vannak a wsd beepitesere?
mennyire jok ezek?


Ebben a cikkben, egy olyan statisztikus gépi fordító modellt mutatunk be, mely kifejezés alapú és a Wikipédiát felhasználva jelentés-egyértelműsítéssel növeli az eredmények pontosságát.