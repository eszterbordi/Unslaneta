\section{A modell definiálása és letező megoldások} \label{sec:model_definition}

\subsection{A statisztikus gépi fordítás}

A statisztikus módszerek használatát a gépi fordításban legelőször Warren Weaver, egy levélben javasolta az 1949-es években. Azonban az elméleti nehézségeknek és az akkori számítógépek fejlettségének tulajdoníthatóan a módszert csak 1990-ben próbálták ki.

Az általunk felhasznált modell -- a kifejezés alapú fordító modell -- a zajos csatorna modellen alapszik -- \ref{fig:modell:fig1}. Tételezzük fel, hogy egy $F$ nyelv valamilyen $f$ szövegének fordítása egy $E$ célnyelvbe az $e$-vel jelölt szöveg. Annak valószínűsége, hogy $f$-et lefórdítva $e$-t kapunk: $P(e|f)$. Felhasználva Bayes képletét a feltételes valószínűség a következő alakban írható:

\begin{equation}
	P(e|f) = \frac{P(e)P(f|e)}{P(f)}
\end{equation}

Az nevezőt elhagyva a következő optimalizálási feladathoz jutunk:

\begin{equation}
	T(f) = argmax_e P(e|f) = argmax_e P(e)P(f|e)
\end{equation} 

Ezzel a feladatunkat három részre osztható: 
\begin{itemize}
	\item
		Nyelvi modell: $P(e)$
	\item
		Fordítási modell: $P(f|e)$
	\item
		Dekóder: $argmax$
\end{itemize}














\begin{figure}[b]
  	\centering
  		\pgfimage[width=0.6\linewidth]{images/smt_translation_model}
  	\caption[smt_translation_model]%
  	{A zajos csatorna modell}
 	\label{fig:modell:fig1}
\end{figure}





\subsection{Jelentés egyértelműsítés}
bemutatni a wsd-t?

\subsection{Létező megoldások}
jelenlegi phrase based megoldasok
	miert jok, miert nem
	osszehasonlitasok mas modszerekkel
milyen probalkozasok vannak a wsd beepitesere?
mennyire jok ezek?