\begin{abstract}
A gépi fordításban jelenleg legelterjedtebb módszer a statisztikus gépi fordítás (SMT). Annak ellenére, hogy ezek a módszerek sokat fejlődtek, továbbra sem tudják megközelíteni az emberek által végzett fordítás minőségét. Ennek egyik oka, hogy nem tudják értelmezni a többjelentésű szavakat, és így ezekben az esetekben hibás eredményt adnak. A szakirodalom aktívan foglalkozik ezzel a témakörrel, ennek eredményeként jöttek létre azok a módszerek, ahol jelentés-egyértelműsítést (WSD) alkalmaznak a helyesebb fordítás elérésére. A saját munkánkban is egy ilyen módszert tárgyalunk. Az általunk kidolgozott WSD-rendszer fő jellemzője, hogy a jelenleg elérhető legnagyobb tudásbázist, a Wikipedia enciklopédiát használja fel. Cikkünkben bemutatjuk a rendszer felépítését, alkalmazását a létező SMT megoldások esetében. A módszert részletes elemzésnek és kiértékelésnek is alávetjük, ezzel igazoljuk annak hatékonyságát.
  
\end{abstract}