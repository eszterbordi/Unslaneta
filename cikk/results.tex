\section{Kísérletek és Eredmények} \label{sec:results}

\subsection{Adathalmazok}

Az \cite{Gabrilovich:2007:CSR}-ben említett Wikipedia alapú adathalmaz\footnote{Concept Wikipedia Adathalmaz (2007),  \url{http://www.lscom.org/datasets}} a népszerű webes enciklopédia 2006. március 26-i állapotát tükrözi. Az XML-formátumú Wikipedia-másolat feldolgozása, utólagos szűrések, valamint nyelvfeldolgozási módszerek alkalmazása után 400,000 fogalmat és 2,800,000 URL-t tartalmazó hierarhia született. 

A OpenSubtitle\footnote{OpenSubtitle, \url{http://www.opensubtitles.org/}} weboldal tulajdonosai teljes adatbázisukat hozzáférhetővé tették, ez 308,000 állományt jelentett, 18,900 film feliratát összesen 59 különböző nyelven (2006. júliusi állapot). Az OPUS filmfelirat-korpusza \cite{Tiedemann:RANLP5} egyirányú párhuzamos korpusz, mely a forrásnyelvi (angol) szövegeket és azok 30 célnyelven meglevő fordításait tartalmazza. A bekezdés-, mondat- vagy szószinten párhuzamosított szövegek igencsak alkalmasak gépi fordítás esetén.

\begin{table}[h!]
\centering
\normalsize
 \begin{tabular}{| c | c |} 
 \hline\hline
 Állományok száma: & 20,400 \\ [1ex]
 
 Tokenek száma: & 149.44M \\ [1ex]
 
  Mondatrészletek száma: & 22.27M  \\ [1ex] 
 \hline
 \end{tabular}
 \caption{Az OPUS tulajdonságai}
\end{table}

A statisztikus gépi fordítás esetén az OPUS adathalmaz került felhasználásra, a Wikipedia alapú adathalmaz pedig a jelentés-egyértelműsítés esetén volt hasznos.

\subsection{Kiértékelési metrikák}

Jelen módszer és létező statisztikus gépi fordítással kapcsolatos algoritmusok összevetése a BLEU score (BiLingual Evaluation Understudy), a WER (Word Error Rate), a METEOR értéke (Metric for Evaluation of Translation with Explicit ORdering), valamint az F-mérték alapján történt.


\subsubsection{BLEU (BiLingual Evaluation Understudy)}
Lényege, hogy a vizsgált rendszer által lefordított mondat kifejezéseit keresi a referenciamondatban. Megoldást kínál a szavak sorrendjéből adódó probléma kezelésére is, hiszen figyelembe veszi a több szóból álló frázisok referenciafordítással való egyezéseit. Minél nagyobb a hasonlóság a két mondat között, annál több pontot kap érte. A BLEU számításának módja formálisan a következőképpen írható le: 
\begin{equation}
BLEU = BP \cdot exp\Big(\sum_{n=1}^N \omega_n \cdot log(p_n)\Big),
\end{equation}
ahol $BP$ (brevity penalty) a rendszer fedését hivatott értékelni, oly módon, hogy lepontozza a referenciafordításnál sokkal rövidebb fordításokat. $p_n$ módosított pontosság,$\omega_n$ az n-gramok súlya (tipikusan 1 értéket vesz fel). Kimutatták róla azonban, hogy különböző elvű (szabályalapú és statisztikai alapú) rendszerek összehasonlítására nem alkalmas, mert az előbbi jobb eredményt szolgáltat még akkor is, amikor a szubjektív megítélés a szabályalapú rendszerrel készített szövegnek kedvez \cite{koehn-monz:2006:WMT}.


\subsubsection{WER (Word Error Rate)}
A WER szófelismerési hiba, a felismerési egységek a szavak (az összetett szó is egy egység). A Levenshtein-távolság továbbgondolása, hiszen már nem fonéma-szinten működik. Ha $S$ a helyettesítések száma, $D$ a törléseké, $I$ a beszúrásoké, $C$ a helyesen felismert, lefordított szavakat jelenti, $N = S+D+C$ pedig a referenciaszöveg szavainak számát adja meg, akkor:
\begin{equation}
WER = \frac{S + D + I}{N}.
\end{equation}


\subsubsection{METEOR (Metric for Evaluation of Translation with Explicit ORdering)}
A METEOR a gépi fordítás kimenetének kiértékelési metrikája, az unigram-pontosság és a visszanyerés (recall) harmonikus közepére alapszik. Szótövesítést, valamint szinonima-találtatást is jelent a pontos szómegfeleltetésen túl. A BLEU hátrányainak kiküszöböléseként vezették be, ám ez is emberi referenciafordításhoz hasonlítja a gépi fordítást.


\subsubsection{F-mérték}
Az F-mérték a pontosság és visszanyerés (recall) értékeinek harmonikus középe. Ezek mértékek az információ visszakeresés körében népszerűek. 
\begin{equation}
F_\alpha = \frac{(\alpha^2 + 1) \cdot PREC \cdot REC}{\alpha^2 \cdot PREC + REC},
\end{equation}
\begin{equation}
PREC = \frac{card(RET \cap REL)}{card(RET)},
\end{equation}
\begin{equation}
REC = \frac{card(RET \cap REL)}{card(REL)},
\end{equation}
ahol $RET$ a visszakapott dokumentumok halmaza, $REL$ pedig a releváns dokumentumoké. Az F-mérték csak akkor vesz fel magas értéket, ha a felidézés és a pontosság mértéke is magas.
\subsection{A kísérlet adatai}





