\section{Eredmények} \label{sec:results}

\subsection{Adathalmazok}

Az \cite{Gabrilovich:2007:CSR}-ben említett Wikipedia alapú adathalmaz\footnote{Concept Wikipedia Adathalmaz (2007),  \url{http://www.lscom.org/datasets}} a népszerű webes enciklopédia 2006. március 26-i állapotát tükrözi. Az XML-formátumú Wikipedia-másolat feldolgozása, utólagos szűrések, valamint nyelvfeldolgozási módszerek alkalmazása után 400,000 fogalmat és 2,800,000 URL-t tartalmazó hierarhia született. 

Az OpenSubtitle\footnote{OpenSubtitle, \url{http://www.opensubtitles.org/}} filmfelirat-korpusza \cite{Tiedemann:RANLP5} egyirányú párhuzamos korpusz, mely a forrásnyelvi (angol) szövegeket és azok 30 célnyelven meglevő fordításait tartalmazza.
Állományok száma: 20,400
Tokenek száma: 149.44M
Mondatrészletek száma: 22.27M



\begin{table}[h!]
\centering
\normalsize
 \begin{tabular}{| c | c |} 
 \hline\hline
 Állományok száma: & 20,400 \\ [1ex]
 
 Tokenek száma: & 149.44M \\ [1ex]
 
  Mondatrészletek száma: & 22.27M  \\ [1ex] 
 \hline
 \end{tabular}
 \caption{Az OpenSubtitle-korpusz tulajdonságai}
\end{table}

